% This is based on the LLNCS.DEM the demonstration file of
% the LaTeX macro package from Springer-Verlag
% for Lecture Notes in Computer Science,
% version 2.4 for LaTeX2e as of 16. April 2010
%
% See http://www.springer.com/computer/lncs/lncs+authors?SGWID=0-40209-0-0-0
% for the full guidelines.
%
\documentclass{llncs}

\begin{document}

\title{Author Identification using SVM}
%
\titlerunning{Hamiltonian Mechanics}  % abbreviated title (for running head)
%                                     also used for the TOC unless
%                                     \toctitle is used
%
\author{Ivar Ekeland\inst{1} \and Roger Temam\inst{2}
Jeffrey Dean \and David Grove \and Craig Chambers \and Kim~B.~Bruce \and
Elsa Bertino}
%
\authorrunning{Ivar Ekeland et al.} % abbreviated author list (for running head)
%
%%%% list of authors for the TOC (use if author list has to be modified)
\tocauthor{Ivar Ekeland, Roger Temam, Jeffrey Dean, David Grove,
Craig Chambers, Kim B. Bruce, and Elisa Bertino}
%
\institute{Princeton University, Princeton NJ 08544, USA,\\
\email{I.Ekeland@princeton.edu},\\ WWW home page:
\texttt{http://users/\homedir iekeland/web/welcome.html}
\and
Universit\'{e} de Paris-Sud,
Laboratoire d'Analyse Num\'{e}rique, B\^{a}timent 425,\\
F-91405 Orsay Cedex, France}

\maketitle              % typeset the title of the contribution

\begin{abstract}
\noindent Author identification is the process, where a text of an unknown author is analysed. It's compared to texts, to which the author is known and a similarity score is calculated. Authorship attribution can be used in areas like law, journalism and forensic linguistics. This paper focuses on solving the Author Identification \cite{stamatatos:2015} task of the PAN 2015 Challenge  http://pan.webis.de 
The described solution uses a machine learning algorithm called `Support vector machine' (SVM) to train and then to evaluate the document corpus.

\keywords{Author identification, forensic linguistics, text mining, machine learning}
\end{abstract}
%

\section{Introduction}

The PAN 2015 author identification task provides a small set of one to ten documents, that are authored by a single person, and a single document with a unknown author. The task is to create a software, that determines whether the questioned document was written by the same person who wrote the known document set.
A training corpus is provided, that contains documents in the specified format. The documents are in four different languages - English, Greek, Spanish and Dutch.  They could be cross-genre and cross-topic. A file with the correct results of the training corpus is provided.

The software should create a file with scores for the corresponding problem, that has to be a real number in [0, 1] inclusive, corresponding to the probability of a positive answer.



%------------------------------------------------

\section{Methods}

\subsection{Overview}

The provided solution uses the software GATE \cite{Cunningham2011a} to annotate the corpus of documents. Groovy scripts are used to extract the annotations in objects. GATE is initialized programaticaly inside a Java project. The extracted objects are analysed using the LIBSVM library \cite{CC01a} 

\subsection {GATE}
The solution uses GATE to annotate and process the annotations of the corpus of documents. A pipeline is used to process the documents. It contains these resources:

\begin{enumerate}
\item Reset PR 
\item ANNIE Tokenizer \cite{Cunningham2002}
\item ANNIE Sentence Splitter
\item Paragraph Transfer
\item Groovy script for adding features
\item Groovy script for adding n-gramms
\end{enumerate}

\subsection {SVM}

SVM is used to create a model using the training data. After that the corpus with the `real' data is evaluated.

\subsection {Features}
To measure the similarity of the text, some features are detected, vectorized and compared using the SVM algorithm.
The features, that we used are:
\begin{enumerate}
\item Average sentence length to character count ratio
\item Average sentence length to word count ratio
\item Average word Length
\item Average paragraph length to word count ratio \cite{sanchez}
\item Average paragraph length to sentence count ratio
\item Punctuation to word count ratio
\item Sentence count to word count ratio
\item Word based n-gramms of sizes 1,2,3 
\end{enumerate}


%------------------------------------------------

\section{Results}



\begin{equation}
\label{eq:emc}
e = mc^2
\end{equation}


%------------------------------------------------

\section{Discussion}

\subsection{Subsection One}


\subsection{Subsection Two}


%
% ---- Bibliography ----
%
\bibliographystyle{plain}
\bibliography{bibliography}

\end{document}
